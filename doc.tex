\documentclass[a4paper,hidelinks]{article}
\usepackage{graphicx}
\usepackage{geometry}
\usepackage{xcolor}

\begin{document}

\newgeometry{top=0mm, bottom=0mm}

~
\vspace{95px}

\begin{center}
{\Huge \textbf{
Rebase
}}\\
\vspace{10px}
\textit{
A crystallization manual for the melting mind
}
\end{center}

\vspace{70px}

\begin{center}
\makebox[\textwidth]{\includegraphics[width=1.45\textwidth]{math_tile_deformation.jpg}}
\end{center}

\thispagestyle{empty}

\newpage

\begin{center}
\vspace*{\fill}
This is not \textit{art}, this is what I literally believe to be true.
\vspace*{\fill}
\end{center}

\restoregeometry
\newgeometry{margin=1.75in}

\pagenumbering{arabic}

\newpage

\begin{center}
\textit{
Screw you guys, I'm going home.  - Eric Cartman
}
\end{center}

\textbf{Matter over mind}.
When someone punches you in the face you will feel pain, no matter how strongly you say \textit{I feel no pain}.
Eliminating negative emotions is more important than creating positive emotions.
Wounds can't heal if you're standing in the middle of the battlefield.
Distance yourself from the source of pain.
If someone's company is making you feel bad, stay away from them.
Stay away from things that make you feel bad.

\newpage

\begin{center}
\textit{
I let go.  Lost in oblivion.  Dark and silent and complete.  I found freedom.  Losing all hope was freedom. - Chuck Palahniuk
}
\end{center}

\textbf{There is no need to do anything}.
There is freedom, to do anything you wish.
Thinking things \textit{should} be done is just social conditioning talking.
There are no rules for life other than: eat, sleep, breathe, and survive.
The rest is made up.
It doesn't matter.
Life is not something that can be wasted.
You can't waste a dance.
Remove everything from your to-do list.
Remove all goals from your life.
Planing is not needed.
Live in the moment.
Surrender.
Stop making yourself do things.
Let things happen naturally.
Remove every effort.
To all that happens, say yes.
Yes.
Let it go.
There is no need to think.
Things can just happen.
Everything you feel bad about, you can also laugh about.
Play for fun, not to win.

\newpage

\begin{center}
\textit{
Realise the fact that you simply "live" and not "live for". - Bruce Lee
}
\end{center}

\textbf{Only do things which you enjoy}.
Concerts are there to be attended.
Movies are there to be watched.
Other people are there to be talked to.
Those activities can fill a life, and that is completely fine.
Every day, give yourself a present.
This way of life is not in any way inferior or superior to any other.
Just be careful not to get addicted.
Take a long, hot shower.
Go for a walk in nature.
Sit quietly and look out of a window.
Your life is your life.
Do things for yourself.
Enjoy life.
Live free or die.

\newpage

\begin{center}
\textit{
No plough stops for the dying man.
}
\end{center}

\textbf{Opinions of others are irrelevant}.
Other people don't spend time thinking about you.
Outside of close friends and family, nobody cares about you.
Notice when you start worrying about what others think.
Notice it and stop giving a fuck.
What do you think?
Only \textit{that} is important.
If you wish to do it, do it.
If others think less of you because of who you are, fuck them.
You can be ok with people not liking you.
If you hurt other people only because they had expectations of you, that's completely their problem.
Instead, if they had an agreement with you, only then is it your problem.
Stop pleasing others.
You failed at something.
So what?
It doesn't count.
Life is not a game you can win or lose.
The same way you can't win or lose a dance.
The main thing people think about is other people.
They compare themselves with them constantly.
It creates a complex of lower value.
Then they start to force themselves to become other people which they do not want to be.
Everyone is copying each other's desires all the time, instead of following their own.
This leads to stupid competitions over stupid prizes.
Think for yourself.

\newpage

\begin{center}
\textit{
There is no spoon.
}
\end{center}

\textbf{Forget everything you think you know}.
I know that I know nothing.
To learn, first, you need to unlearn.
You got into chaos because your model of the world was wrong.
Even if your model is almost completely correct, there are cracks in it.
There are imperfections, and going into those blind spots, those unknown unknowns, you can acquire knowledge that would completely change your view of things.
The map is not the territory.
Your interpretation of reality is not reality.
You are not obliged to believe your negative thoughts.
The problems you think you have are an illusion.
The process of healing is the process of disintegrating illusions.
The feeling of loss.
It is an illusion.
It's ok that right now you feel that way.
There is nothing to be gained from ignoring it or trying to block it.
Still, keep in mind that it's an illusion.
You never lost anything because you never had anything.
You never even needed anything.
You will "lose" everything again.
And you will be ok with it.
Life goes on, and the truth changes.
What was once true is often no longer true just a little while later.
To grow, people have to let go of the principles and standards with which they define themselves.
You are not who you thought you were.
Being wrong feels the same way as being right does.
Remember that the mind doesn't like to change itself in the face of new information.

\newpage

\begin{center}
\textit{
Do you know why the cup is useful? Because it is empty. - Bruce Lee
}
\end{center}

\textbf{Assumptions are not knowledge}.
When events happen, we form an interpretation of them.
Of what people meant when they said something.
Of what is going to happen.
Of what people think about various things.
Almost all of those interpretations are an illusion.
The truth is not what you think it is.
As you live, your interpretations accumulate.
Keep your mind clean of garbage interpretations.
Think step by step.
Pay close attention to the distinction between knowledge and assumptions.
Create a great number of alternative hypotheses for why something happened.
When you feel something is true ask yourself: is perhaps the opposite true?
Relax your expectations when going into things.
You do not \textit{know} if things are going to go well or not.

\newpage

\begin{center}
\textit{
We are not special. We are not crap or trash, either. We just are. We just are, and what happens just happens. - Chuck Palahniuk
}
\end{center}

\textbf{There are no wrong feelings}.
It's as if someone put a match on you and you feel hot.
Of course you feel that way.
You are free to feel that way, that's ok, don't worry about it.
And if you say further \textit{but I can't help worrying about it}, then ok, worry about it.
Go along with it.
You don't know what you're supposed to do.
What can you do?
If you don't know what you're supposed to do, you watch.
Watch not only what's going on on the outside but also what's going on on the inside.
Treat your own thoughts, reactions, and emotions as if those inside reactions were also outside.
And you're just watching them.
Without attempting to change it in any way.
Without judging it.
Without calling it good or bad.
When some emotion feels overwhelming, remember that it too will pass.
The subconscious and the conscious part of your brain wish to communicate.
The rational part of you can't hold the emotional part in submission.
This break in communication comes from the rigid idea that things \textit{should} be a certain way, that you \textit{should} feel this or the other.
Drop your assumptions.
The subconscious needs to express itself.
Blocking your emotions blocks learning.

\newpage

\begin{center}
\textit{
The fundamental delusion: there is something out there that will make me happy and fulfilled forever. - Naval Ravikant
}
\end{center}

\textbf{The only thing which exists is now}.
The past and the future are just mental constructs, they are illusions.
Thoughts of what happened in the past.
And how it could have gone better than it did.
Or, how it was better than the state you are in right now.
Thoughts about pursuing a future.
Thoughts of making one's happiness depend on something which isn't here at all.
When you fulfill your desire, another one will take its place.
Nothing will make you happy forever.
Other people will not save you, nor will they make you happy forever.
The present is the only place you live in.
Memory is a tool.
The reason we have memory is to learn from experience.
What we learn are models.
Once we have the models, the training data can be discarded.
If you remember that something bad happened, and can figure out why, then you can try to avoid that thing from happening again.
That's the purpose of memory.
It’s not \textit{to remember the past}.
It’s to stop the same thing from happening over and over.
The longest-lived and those who will die soonest lose the same thing.
The present is all that they can give up, since that is all you have, and what you do not have, you cannot lose.

\newpage

\textbf{You are not alone}.
There are a lot of similar people to you out there, same as there have been before in history.
They walked the same path you walk.
They had the same thoughts you have.
Good people are out there.
The world is a large place.
Everything could be ok in the end.
Not all is lost.

\newpage

\begin{center}
\textit{
First you have to give up. First you have to know, not fear, know - that some day you’re gonna die. - Chuck Palahniuk
}
\end{center}

\textbf{Everything is exactly as it should be}.
You are not a victim.
You can't blame others for how you feel.
You have your own path.
Whatever happened to you, you have personally contributed to it.
And that was not a mistake.
You do not need anything from other people.
You do not \textit{deserve} anything.
You do not \textit{undeserve} anything.
It has nothing to do with deserving.
Complaining is useless.
You think others are underestimating you.
They are estimating you the best they can, given how little they see.
Work but do not control.
Create but do not possess.
Succeed but do not dwell on success.
Achieve without arrogance.
Rise without domination.
Yield and remain whole.
Bend and remain straight.
Be worn out but become renewed.
Have little and receive.
Those who boast about themselves do not last.
Those who stand on tiptoes do not stand firmly.
Those who rush ahead don't get very far.
Those who try to outshine others dim their own light.
Return to the state of the infant.
A good commander achieves result, then stops.
And does not dare to reach for domination.
Achieves result but does not brag.
Achieves result but is not arrogant.
Achieves result but only out of necessity.
The ultimate honor is no honor.
Do not wish to be shiny like jade, be dull like rocks.
Close the mouth.
Shut the doors.
Blunt the sharpness.
Unravel the knots.
Mix the dust.
Learn to laugh at yourself.
Remember that you are going to die.
In fact, you are dying all the time.
You have nothing to lose.
You are already naked.
There is no reason not to follow your heart.

\newpage

\begin{center}
\textit{
Sometimes I go about in pity for myself, and all the while, a great wind carries me across the sky. - Ojibwe saying
}
\end{center}

\textbf{Do unto yourself as you would do unto others.}
Look at yourself from the outside perspective.
Do not let your critical voice say anything to you that you wouldn’t say to someone you care about.
Do not punish yourself in a way that you wouldn’t punish someone you care about.
Imagine how your life would look to an outside observer who was with you all along.
All the obstacles you had, and you overcame.
How you were in pain.
Pause and say to yourself:
\textit{
Here I am.
I am in a difficult situation.
I am struggling and I am in pain.
}
And then, react to this information just as if we were hearing it from a close friend.
Say to yourself:
\textit{
I am so sorry.
I want to care for you in the best way that I can.
How can I help you? What can I do for you?
}
And then we must answer these questions for ourselves.
If a member of your family had the same problems you now have, you would hug them.
You would be kind to them.
Now you may think you hate yourself.
On some deep level, perhaps you do.
Going even deeper you will see, underneath it all, underneath the sea of disappointment, anger, horror, and sadness, if you look hard enough, you will find that you love yourself.
You have suffered enough.
Suffering is boring.
You want yourself to be at peace.
Say to yourself:
\textit{
May I accept my pain, without thinking it makes me bad.
May I be at ease and happy.
}

\newpage

\begin{center}
\textit{
I am a human being, therefore nothing human is foreign to me. - Terence
}
\end{center}

\noindent
\textbf{You are an animal}.\\
Inside all of us, there is darkness.\\
Our minds are evolved.\\
Deep parts of our brain are primitive.\\
Since we are born, we are selfish.\\
Our emotions make no sense.\\
Our head is full of lies.\\
Our ability to reason is limited.\\
Yet, there is a flame within us.\\
We wish to know why the stars shine.\\
Our heaven is a selfless union with another.\\
We are ugliness, seeking beauty.\\
Imperfections can be endearing.\\
Not just with romantic partners but with everyone.\\
Not just with people but with art and work.\\
Not just with others but with yourself.\\

\newpage

\begin{center}
\textit{
The most terrifying thing is to accept oneself completely. - C.G. Jung
}
\end{center}

\textbf{Stop trying to be someone else}.
Let go of who you think you should be.
Let go of who others think you should be.
Let your heart and your intuition guide you.
As you do that, the real you emerges.
Sink into who you are.
When you think something is wrong with you, you turn to the outside world to tell you what is wrong.
They are happy to provide you with bullshit about what you should change.
Instead, look within at who you are, and accept that.
It is not better to act than to be still.
It is not better to be \textit{productive} than to do nothing.
It is not better to be disciplined than it is to be relaxed.
What is a weakness in one context can be a strength in a different context.
Ego is a good thing, it is condensed experience, processed and compressed information.
Walk with who you are.

\newpage

\begin{center}
\textit{
I say never be complete, I say stop being perfect, I say lets evolve, let the chips fall where they may. - Chuck Palahniuk
}
\end{center}

\textbf{You will never be perfect}.
You have done something you think you shouldn't have.
You have failed to do something you think you should have.
This does not make you a bad, evil, or immoral person.
It makes you a normal person.
The event which occurred is not completely your fault.
You had your share of influence, which was not so large as you think.
The person that you feel that you wronged may feel completely differently about the situation.
That person has a lot of other things in their life.
Those things make them feel very different than you do, in a lot of ways.
You think you should have done differently because a perfect person would do differently.
That is not in line with reality.
You're not supposed to be some imaginary perfect person.
You're supposed to be you.
You could not predict all outcomes of your actions.
You did the best you could given the situation and the knowledge you had at that moment.
In the moments of a catastrophe unfolding, there is almost no control over what's happening.
There are almost no thoughts, it's like everything is on auto-pilot.
Because it is.
Do you know why you did what you did?
Understanding leads to forgiveness.
Forgiveness allows you to learn from the past.
Learning leads to change.
Your sins are not a part of yourself.
Flow with the changes.
Everything that will happen, must happen, can not \textit{not happen}.

\newpage

\begin{center}
\textit{
I believe that it is better to tell the truth than to lie. I believe that it is better to be free than to be a slave. And I believe that it is better to know than be ignorant. - H. L. Mencken
}
\end{center}

\textbf{Open up}.
Share your thoughts with someone close.
Seeing the world through the eyes of others will make you view your life in a new light.
When you feel an urge to say something to close friend, don't be afraid.
It's better to speak than not to speak.
Lying it is the major source of all human stress.
The kind of lying that is most deadly is withholding.
Keeping back information from someone we think would be affected by it.
Say what you think, speak the truth.
What can be destroyed by truth should be destroyed by truth.
What is true is already so.
Owning up to it doesn't make it worse.
Not being open about it doesn't make it go away.

\newpage

\begin{center}
\textit{
If I commit to doing something, then I commit to doing it right now. - Naval Ravikant
}
\end{center}

\textbf{Make your life a tiny bit easier and more enjoyable}.
Day by day, step by step.
If something bothers you, fix it.
Instead of thinking about things you can't change, focus on your immediate environment.
Improve small everyday things.
Clean your room.
Do things that are easy to do.
Play with the task.
Learn to love it.
Even if you don't want to work on a task, you can make some preparations, so once you \textit{really do} work on it, it will be easier.
Knowledge of the future is impossible.
The only thing you can commit to doing is what you can do right now.
Try to do something, however small it is, every day.

\newpage

\begin{center}
\textit{
Only after disaster can we be resurrected. It's only after you've lost everything that you're free to do anything. Nothing is static, everything is evolving, everything is falling apart. - Chuck Palahniuk
}
\end{center}

\textbf{Discover something you enjoy doing}.
We need a lot of downtime.
A lot of daydreaming time.
A lot of slack and unstructured time.
We need boredom.
We need to be out of the limelight for long periods of time.
Only failure gives you that kind of isolation downtime where you can find yourself.
You can find inspiration in art.
You can do things just for fun, just to see what will happen.
Go outside, walk around, go into nature.
Follow your curiosity.
Approach things like a scientist.
The world is an interesting place which you can study.
Experiments can be done even if you are feeling unwell.
Moving in different directions, you will see what you like.
Some things nurture a certain aesthetic.
Watch in which direction you are naturally flowing.
When it comes to life, gardening works better than engineering.
Discover your likes and dislikes.
Let your intuition guide you.
Take feedback from your environment.
Creating is more rewarding than consuming.
You can produce something; not because you have to, but just for fun.
Mess things up.
Improvise.
Do silly, unconventional, spontaneous, and irrational things.
What would you do if you had an infinite amount of resources and an infinite amount of time?
Whatever is within you, express it.
Release it.
Do whatever \textit{you} intrinsically \textit{wish} to do.

\newpage

\begin{center}
\textit{
Discipline is just you fighting with yourself to do something you don't want to do. Find things that excite you. - Naval Ravikant
}
\end{center}

\textbf{A horse which loves the track runs faster}.
That is a horse which, when exhausted, stops running.
Lack of motivation for some action could be your subconsciousness telling you that the goal is not something you really want.
Or maybe the action you are considering is not a good way to achieve the goal.
Or maybe the goal is not something that should be at the top of your priorities right now.
A great way to waste time is to achieve a goal, only to realize you did not truly want it.
When you are thirsty, it takes no energy to drink water.
It takes energy \textit{not} to drink water.
Pushing yourself is not needed.
You need to feel the pull of the thing calling you.
There is no need for pain.
If you have to \textit{try} to care about something, or \textit{try} to want something, perhaps you don't care about it and perhaps you don't want it.
In that case, don't do it.
If the thought of \textit{not} doing it hurts more than the thought of going through the process; if the thought of having never tried it at all terrifies you; then do it.
When you like swimming, you don't swim just to get to the other side.
You do it to move, to feel the water.

\newpage

\begin{center}
\textit{
School, politics, sports, and games train us to compete against others. True rewards such as knowledge, love, and equanimity, they come from ignoring others. - Naval Ravikant
}
\end{center}

\textbf{Ignore irrelevant things}.
Take one thing you care about the most and go all the way.
Remove all barriers.
Give zero fucks about other things.
You got to shut one door before another one can open.
Caring about many things at the same time creates anxiety.
You can change what you care about later.
Now focus on the most important thing in this phase of your life.
About other things, things you don't give a fuck about, be spontaneous.
Intelligence is ignoring what needs to be ignored.

\newpage

\begin{center}
\textit{
Our business is not to see what lies dimly at a distance, but to do what lies clearly at hand. - Bruce Lee
}
\end{center}

\textbf{Your every action should be beneficial in and of itself}.
Stop planning.
Solving beats planning.
Think short-term.
The shortest path between two points is a straight line.
Things that look hard or boring from a distance, when you come closer and apply action, get transformed.
Do more, think less.
You learn by doing.
To become good at something a lot of repetition is needed, practice, experience.
Move step by step.
What is needed is a lot of specific knowledge, knowledge of the details, doing things that don't generalize, which don't scale, rote memorization.
Fail!
Failure is the source of knowledge.
Ignore theory.
Experience is the only way to learn.
Always go with the line of the least resistance.
Act without effort.
When you feel inspired, do it, don't pause too long.
Inspiration is perishable.
Use it while you have it.

\newpage

\begin{center}
\textit{
Nothing that we do lasts. Eventually, you will fade. Your works will fade. Your children will fade. Your thoughts will fade. This planet will fade. The sun will fade. It will all be gone. - Naval Ravikant
}
\end{center}

Every endeavor is a sequence of small tasks.
Take small steps.
Plan difficult tasks through the simplest tasks.
Complete them one step at a time.
Achieve large tasks through the smallest tasks.
You may think the tasks are larger than yourself.
They are in fact smaller than yourself.
You may think you have not been doing enough.
This does not make you a lazy person or a procrastinator.
In fact, you have been doing exactly as much as you should have.
You have been doing enough.
Life is a vast ocean of neverending problems.
The only question is: \textbf{which problems do you wish to have}?

\newpage

\begin{center}
\textit{
Empty your mind, be formless, shapeless, like water. If you put water into a cup, it becomes the cup. You put water into a bottle and it becomes the bottle. You put it in a teapot it becomes the teapot. - Bruce Lee
}
\end{center}

\textbf{Instead of addictions, be useless}.
Notice when a behavior becomes an addiction.
Sometimes you do things because you wish to do it.
Some other times you do it automatically, without even thinking.
Performing an activity you are addicted to is not a pleasant experience.
Those activities don't have any value.
They block creativity.
Addictions are hijacking your brain and taking your freedom away.
Try to notice it and instead meditate, or do nothing.
Remove the temptations from your environment.
A tree that is useless for wood carving is the one that survives and offers shade.
The best ideas sometimes come in the shower, or during a walk.
Being bored gives you time to think.
To wander into uncharted territory.
To think about what's happening.
Lets curiosity develop.
Gives you ideas for experimentation.
Gives you time to flow.
The best meditation is doing nothing.

\newpage

\begin{center}
\textit{
The death-grip with which one holds on to principles is a source of unhappiness and anger. - Brad Blantnon
}
\end{center}

\textbf{Anger is the result of incorrect predictions}.
If you knew what was coming all along, you would not be angry.
You think that things \textit{should} be this way or that way.
Question what you are saying to yourself.
You think it's not fair.
You think the way things are is not just.
You think things should be different.
Well, they shouldn't.
The universe is as it is.
It is your picture of the universe that is wrong.
Your expectations were wrong.

\newpage

\begin{center}
\textit{
Do not seek revenge or bear a grudge against anyone, but love your neighbor as yourself.
}
\end{center}

\textbf{We are all children}.
Most of the bad behavior of other people comes from fear, not malice.
Hating others leads to hating yourself.
Every human being is a combination of positive and negative attributes.
The attribute of others you are judging is also, to a certain extent, present in you.
Ask yourself if the other person had the right to act in such a way.
In most cases, they did have the right.
It's a free country.
You don't actually \textit{know} why the other person did what she did.
It's never too late to forgive.
When you love someone, they can't hurt you.
Love is a shield.

\newpage

\begin{center}
\textit{
Much unhappiness has come into the world because of things left unsaid.
}
\end{center}

\textbf{Resentment tears relationships apart.}
Sometimes, changing the way you think can change how you feel.
Other times, you can't change how you feel, the original emotion is still there, even though you think it's irrational to feel that way.
In that case, just feel whatever it is that you feel.
Repression of emotions often just hides them, it does not destroy them.
They are there in the background, looking for justifiable ways to express themselves, for something they can latch on to.
Then they slowly start coming out in various small ways, over the years.
The ugly emotions which we are afraid to even experience are repressed the most.
There are emotions and thoughts you just want to bury.
Allow yourself to feel those emotions in their full intensity.
Communicate those emotions to other people.
Other people can't read your mind and automatically know what you feel or think.
Create some environment in which you can be completely honest with others.
The first step is describing the facts, without judging anyone.
What do you see?
The second step is to explain how that made you feel.
The third step is to express an alternative way things could be done.
Some problems can't be solved.
In case those problems aren't a big deal, you can let them be.
In case they are a big deal and truly can't be solved, only then walk away.

\newpage

\begin{center}
\textit{
The trouble with most of us is that we'd rather be ruined by praise than saved by criticism. - Norman Vincent Peale
}
\end{center}

\textbf{When someone is criticizing you, it is an opportunity}.
The critic may be right.
The first step is to ask questions without judgment.
Attempt to see the world through the critic's eyes.
Attacking the critic leads to destruction.
Retreating from the critic leads to humiliation.
Instead, stand your ground and disarm.
Find some way to agree with the critic.
Perhaps you agree in principle, but not in the specifics of the situation.
Express your point of view with an acknowledgment you might be wrong.
Make the conflict one based on fact rather than personality or pride.
In many cases, you will be just plain wrong, and the critic will be right.
In such a situation your critic's respect for you will increase if you agree with the criticism, thank the person for providing you with the information, and apologize for any hurt you might have caused.

\newpage

\begin{center}
\textit{
Stand up straight with your shoulders back. - Jordan Peterson
}
\end{center}

\textbf{No one is allowed to fuck with you}.
People naturally start to push boundaries, the same way children do.
When someone is treating you badly, be honest and tell them what's bothering you.
Refuse to be pushed around.
Ask them why are they doing the thing they are doing.
No one is allowed to treat you like you're a loser.
Piling up resentment will push you to the dark side.
Your dark side is a serious matter.
Any anger that is not coming out, flowing freely, will turn into sadism.
When someone is doing something you don't like, tell them.
Tell the truth and let the chips fall where they may.
Dare to be dangerous, in a controlled way.
So that you don't become dangerous in an uncontrolled way.

\newpage

\begin{center}
\textit{
If you're not doing what you want, if you're not earning, you're not learning, what the fuck are you doing? Don't spend time making other people happy, other people being happy is their problem, it's not your problem. - Naval Ravikant
}
\end{center}

\textbf{You are doing it for yourself}.
You are personally responsible for your continued existence.
You need to find a path for yourself.
You seek knowledge for yourself.
The man is an end into himself.
When you love someone you wish to help them and are helping them out of your free will.
You are doing it because \textit{you} wish to, not because you are sacrificing yourself for others.
The path of unwilling sacrifice leads to suffering, resentment and hate.
No more suffering.
You have sufferend enough.
If you are suffering, something is very wrong.
If you are healthy and enjoying life, it will open you up to have a positive impact on the people around you.
Your health and happiness will have a ripple effect.
When you are healthy and strong, you’re freed up to give to others.
To really experience your life, the focus must be on you as the center of the perceived universe.
Any time you're doing something that you think is not for you, examine both your thinking and your actions.
If it isn't for you, you’re doing it wrong.
Ask for what you want.

\newpage

\begin{center}
\textit{
It is difficult to find happiness within oneself, but it is impossible to find it anywhere else. - Arthur Schopenhauer
}
\end{center}

\textbf{Isolation is a gift}.
We are all alone.
We shall all someday look back on our lives and see that, despite the company of others, we were alone the whole way.
Not lonely, but essentially, and finally, alone.
This is what makes your self-respect so important.
There is no way you can respect yourself if you must look in the hearts and minds of others for your happiness.
The Tao which can be named is not the eternal Tao.
In solitude, the nameless can be found.
We seek beauty and truth, which can not be fulfilled by others.
Trying to do so results in a godlike idealization of the partner and dependence on them for our self-worth.
Nobody is going to save you.
Only alone you can find yourself.
Solitude is for the mind as food is for the body.

\newpage

\begin{center}
\textit{
You must have chaos within you to give birth to a dancing star. - Friedrich Nietzsche
}
\end{center}

\textbf{Remember what you value}.
Take a cosmic perspective.
Everyone spends so much time looking around themselves.
Almost no one is looking up.
Let your principles and ideals be at the forefront of your mind.
What is constantly tugging at you and removing your focus?
It is not a matter of what is important, or what is productive.
It is a question of do you wish to spend time like that.
You have a choice of what to think about.
What you pay attention to.
There are two fundamental ways of being: flow and discomfort.
What's obstructing your flow, my friend?
There are two transitions: out of flow and into flow.
Out of flow transition cannot be controlled since when you are in flow you are lost in it.
Into flow transition can be helped by paying attention to what you are thinking about and then deciding what to do about the thought.
You may decide to think about it later.
No need to write it down, if it's important enough it will reappear by itself.
Or you may decide that it's not worth thinking about at all.
Or you may decide to pay attention to it.
In that case, look at the thing, slowly focus on every single detail of the experience.

\newpage

\begin{center}
\textit{
Unlimited possiblities are not suited to man; if they existed, his life would only dissolve in the boundless. - The I Ching or Book of Changes, Hexagram 60
}
\end{center}

\textbf{Obtain an organizing idea}.
An ultimate ideal to play towards.
It should never be achievable, an infinite game.
It should not depend on others, only on yourself.
An organizing idea is \textit{logic}, everything else is \textit{probability}.
In order to play, you need to create time and space for play.
Thus, limits are needed.
The spacetime you carve out for play will be attacked.
To gain money, power, status, or any other goal, play must be sacrificed.
The goal is always in the future, while life is always here, in this moment.
The limits of play spacetime need to be defended.
This choice needs to be made each day anew.
Resist addiction.
Obtain flow.
Never forget that it's a game.
An infinite game, played for fun.

\newpage

\begin{center}
\textit{
Because it's there. - George Mallory
}
\end{center}

\textbf{True value is intangible}.
It can't be quantified.
A flash of lightning in the dark!
You can be grateful you witnessed magic.
Set up the conditions in which magic can be produced.
In isolation and silence, wait and the muse shall deliver.
Art is the only way to reach truly great states of mind.
Either by appreciating art or by creating it.
The art that we \textit{shall} make will have only God as its intended audience, and all other beholders will be merely incidental.
This is how it must be, and how it always has been, with regard to great art.
The production of art began with ceremonial objects destined to serve in a cult.
The sheer existence of such objects was always more important than their display.
The elk portrayed by the man of the Stone Age on the walls of his cave was an instrument of magic.
It was meant for the spirits.
Certain sculptures on medieval cathedrals are invisible to the spectator on ground level.
These objects are aimed at Heaven.

\newpage

{\setlength\parindent{0pt}
\textbf{Christopher}: \textit{You ever feel like nothing good was ever gonna happen to you?}
\textbf{Paulie}: \textit{Yeah, and nothing did. So what?}
}
\newline
\newline
Rivers and oceans can be the kings of a hundred valleys.
Because they stay low.
While alive, the body is soft and flexible.
When dead, it is hard and rigid.
All living things, grass, and trees, while alive, are soft and malleable; when dead, become dry and brittle.
Thus that which is hard and stiff is the follower of death.
That which is soft and yielding is the follower of life.
Therefore, an inflexible army will not win.
The arrogant, competitive, ruthless, judgmental, vengeful, eager to prove themselves...
They occupy a lowly position; while the soft occupy a higher place.

\newpage

\begin{center}
\textit{
One of the symptoms of an approaching nervous breakdown is the belief that one's work is terribly important. - Bertrand Russell
}
\end{center}

\textbf{Keep your identity small}.
We have a capacity to identify strongly with a particular mission.
We invest so much of ourselves into an effort that we are unable to separate ourselves from it.
The failure of the effort becomes identical, in our mind, with our own loss of identity.
Cutting your losses becomes cutting your own throat.
We become unable to separate losing from being wrong, being wrong from social death, and social death from actual death.
If you're not a person winning this hill, what are you?
Well, you are a not dead yet person, which is everything, even if it seems like nothing at the moment.
Know your limits and make sure you do not cross the line that separates indefinite survivability from a spiral of self-destruction.

\newpage

\begin{center}
\textit{
This is water. - David Foster Wallace
}
\end{center}

\textbf{Everybody worships}.
The only choice we get is what to worship.
The compelling reason for choosing a spiritual thing to worship is that anything else you worship will eat you alive.
If you worship money, then you will never feel you have enough.
Worship your body and beauty and you will always feel ugly.
On one level, we all know this stuff already.
The trick is keeping the truth up front in daily consciousness.
Worship power, you will end up feeling weak and afraid, and you will need ever more power over others to numb you to your own fear.
Worship your intellect, being seen as smart, you will end up feeling stupid, a fraud, always on the verge of being found out.
But the insidious thing about these forms of worship is not that they’re evil or sinful, it’s that they’re unconscious.
They are default settings.
And the so-called real world will not discourage you from operating on your default settings, because the so-called real world of men and money and power hums merrily along in a pool of fear and anger and frustration and craving and worship of self.
The really important kind of freedom involves attention, and awareness, and discipline, and being able to choose what to think about.
The alternative is unconsciousness, the default setting, the rat race, the constant gnawing sense of having had, and lost, some infinite thing.

\newpage

\begin{center}
\textit{
Life is a series of natural and spontaneous changes. Don't resist them; that only creates sorrow. Let reality be reality. Let things flow naturally forward in whatever way they like. - Lao Tzu
}
\end{center}

\textbf{You are a process of change}.
Awareness of the immediate moment-to-moment passing of the world, the ever-changing existence, the fragility of our own being, and the relative unimportance of the personality we think we are, is a terrifying experience.
It feels like dying.
Instead of avoiding the trauma of the realization, go into it.
How much of you has already died.
There were whole worlds, of colors, sounds and scents, unique elaborate scenes and emotions, some real, some only in your mind’s eye...
They could not even be described by words, and now they're gone, forever.
Your cells renew themselves periodically, old concepts are replaced by new and your new being reflects life’s experience.
Flow with the changes.
Be that new being.
You don’t have to make the person-you-used-to-be happy.
Adapt.
Be like water.
The only rule to life is that there are no rules.
Invent yourself and reinvent yourself.
Life is a continuing process of change.
Resistance to change and personal growth is pain.
Flow with the changes that are happening to you and enjoy the unfolding of your own life.

% What we do in life, echoes in eternity.

% Ultimately, we are all dead men. Sadly, we cannot choose how, but we can decide how we meet that end: not whining and complaining.

% Ask, and it shall be given you; seek, and ye shall find; knock, and it shall be opened unto you. For every one that asketh receiveth; and he that seeketh findeth; and to him that knocketh it shall be opened.

% We love ourselves more than other people, but for some reason we care about other people's opinions more than our own. That's insanity.
% Do everything as if it was the action of a dying person.
% You are not stuck. You don't control what happened. You control how you respond. You always have a choice. Another path is always open.
% Just take the first step. Forget about the other steps, just the first.
% Stop reading email or any other messages from other people in the morning. You need to control the inputs which are coming into you. Start the day with a meditation. Pay attention to your body, to what you see, to what you hear, as if you are feeling it, seeing it and hearing it for the first time. Think about the day ahead of you, or think about anything, without distractions. Then, after you choose the next action, do it. After your morning has started in the right way, the rest of the day may as well. Optimize your morning first, then optimize the moments after it.
% Is this essential? Only put essential things on your todo list. If it's not essential, ignore it.

% “Consider the lilies of the field, how they grow. They toil not, neither do they spin, and yet I say unto you that even Solomon in all his glory was not arrayed like one of these. Take therefore no thought for the morrow, for the morrow shall take thought for the things of itself.”
% If it is not up to you, ignore it. Worrying does not help. We suffer more in imagination than in reality. It adds up cumulatively to more suffering. Prepare if you can and if it is rational to prepare. Then, forget about it.

% This isn't something I have to do, it's something I get to do.
% This isn't something that happened to me, it happened for me.
% Time is the most precious resource.
% You have the power to have no opinion.
% There is something good in everyone.
% Never be overheard complaining, even to yourself. Complaints solve nothing.
% The best revenge is to be unlike him who performed the injury.
% Put your initial impression to a test. Your first impression may be right, or it may be wrong. Do not react emotionally in a moment. Think, then react.
% Events are objective, our interpretations of them are not.
% Don't be afraid to ask for help. You're like a soldier charging a wall, if you need to reach up for a comrade to help you up, so what?
% Your job is to change yourself, not other people.
% Forgiveness is a gift you give to yourself.
% Find beauty in the ordinary things. todo aurelius lion's furrowed brow quote
% Be the rock the waves are crashing into.
% Anger: you can let it go.



% “Ask and it shall be given you, seek and ye shall find, knock and it shall be opened unto you. For everyone that asketh receiveth, and he that seeketh findeth, and to him that knocketh, it shall be opened.”
% This life is a sound of a one hand clapping. This world, these thoughts, these experiences and feelings, they are beyond absurd. In this infinite absurdity it makes the same amount of sense to choose to be unhappy as it does to choose to be happy. Mu. Now you can go fuck yourself. Amen amen amen amen.



% “you say that it is impossible to lay any plans for the future until you are sure you have a future. I say Nonsense! None of these things matter. If you expect a future you must carve it out in the face of these things.”
% Hunter S. Thompson

% Are your youthful dreams of adventure, accomplishment, travel and romance buried under the cloak of conformity?

% “Who is the happier man, he who has braved the storm of life and lived or he who has stayed securely on shore and merely existed?”
% Hunter S. Thompson

% there's always something you can do: introduce order. Or, if there's too much order in your life: introduce chaos.

% todo
% https://www.youtube.com/watch?v=TfBKsatedSU

% Sparta, Rome, The Knights of Europe, the Samurai. They worship strength, because it is strength that makes all other values possible. - Han, Enter the Dragon

% instead of forcing yourself to behave a certain way, or think a certain way, let yourself manifest and learn about yourself... who are you? what do you like? why do you do the things you do?
% you can read as many books as you want, but if you can't read yourself you will never make any progress
% compartmentalizing is ok
% self-sabotage brings a reward, it has some perceived benefit to you (alongside with an obvious cost)
% change often feels uncomfortable
% stop blaming other people. stop blaming society. you are living in an ok society. other people were ok to you. the same as you take ownership at work, take it at home.
% have uncomfortable conversations sooner

% Those thoughts and ideas, you had two months ago, are expired. You are a new you. Don't feel guilty to be free.

\end{document}
